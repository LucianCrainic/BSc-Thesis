% ============================================== %
%
% Introduction Chapter
%
%% ============================================== %
\hypersetup{colorlinks=true, linkcolor=red}
\chapter{Introduction} \label{chap:introduction}

    This thesis is structured in the following way: \textbf{Chapter} \ref{chap:introduction} presents the problem statement, literature review, dataset overview and the aims and objectives of the study. \textbf{Chapter} \ref{chap:dataset_analysis} presents the data collection methodology, data structure and attributes, participants characteristics, movements visualization and data processing. \textbf{Chapter} \ref{chap:methodology} presents the methodology used in this study, including the models used, data splitting and feature engineering. \textbf{Chapter} \ref{chap:results_and_discussion} presents the evaluation metrics used, the results obtained and the discussion of the results. Finally, \textbf{Chapter} \ref{chap:conclusions} presents the conclusions of this study and future work.

   \section{Problem statement}

      Traditionally, movement classification requires high quality sensors and and complicated computer vision algorithms. However, with the advent of the Microsoft Kinect sensor and the release of the Kinect SDK \cite{jana_kinect_2012}, it is now possible to obtain high quality 3D skeleton data with a relatively low cost device and with minimal effort. This opens up the possibility of using this data to classify movements performed by individuals, which can be used in a variety of applications, such as rehabilitation, sports, and fall risk assessment. In this thesis, the focus is on the latter, with the goal of using Kinect skeleton data to classify movements performed by elderly individuals.
  
   \section{Literature review}
      
      \todo[inline]{Overview of the literature on the topic of movement classification using Kinect data. Organize the papers you want to review and tell for every paper what is the main contribution and how it relates to your work.}

   \section{Dataset overview}
      
      In this thesis, a dataset of Kinect skeleton data was used, provided by the PsyComp Lab. The dataset is composed of recorded movements performed by a group of 22 individuals. The movements were performed in front of a Kinect sensor, which recorded the movements and saved them as a series of 3D coordinates. The dataset contains 10 different movements, each performed a various number of times by each individual. The movements are listed in Table \ref{tab:movement_table}.
       
      \begin{table}[ht]
         \centering
         \caption{Movements used in this study, along with a brief description.}
         \label{tab:movement_table}
         \begin{tabular}{@{}ccc@{}}
            \toprule
            \textbf{No.} & \textbf{Movement Name} & \textbf{Description} \\
            \midrule
            1 & Reach Overhead & \begin{tabular}[t]{@{}p{9cm}@{}} In a standing position, the subject raises one of their arms above their head.\end{tabular} \\
            
            2 & Chair to Chair & \begin{tabular}[t]{@{}p{9cm}@{}} Starting from a sitting position, the subject stands up, then sits down on another chair.\end{tabular} \\
            
            3 & Cross-Reach Left & \begin{tabular}[t]{@{}p{9cm}@{}} In a standing position, the subject using their left arm reaches across their body to the right side.\end{tabular} \\
            
            4 & Cross-Reach Right & \begin{tabular}[t]{@{}p{9cm}@{}} In a standing position, the subject using their right arm reaches across their body to the left side.\end{tabular} \\
            
            5 & Reach Forward & \begin{tabular}[t]{@{}p{9cm}@{}} In a standing position, the subject reachs forward with one of their arms.\end{tabular} \\
            
            6 & Hoop Walk & \begin{tabular}[t]{@{}p{9cm}@{}} Starting from a standing position, the subjects walks inside a hoop placed on the floor and then walks out of it.\end{tabular} \\
            
            7 & Right Leg Stand & \begin{tabular}[t]{@{}p{9cm}@{}} In a standing position, the subject raises their left leg and holds it in the air for a few seconds.\end{tabular} \\
            
            8 & Left Leg Stand & \begin{tabular}[t]{@{}p{9cm}@{}} In a standing position, the subject raises their right leg and holds it in the air for a few seconds.\end{tabular} \\
            
            9 & Mat Walk & \begin{tabular}[t]{@{}p{9cm}@{}} Starting from a standing position, the subject walks over a mat placed on the floor and then off it.\end{tabular} \\
            
            10 & TUG Walk & \begin{tabular}[t]{@{}p{9cm}@{}} Starting from a sitting position, the subject is asked to stand up, walk 3 meters, turn around, walk back to the chair and sit down while being timed.\end{tabular} \\
            \bottomrule
         \end{tabular}
   \end{table}

   \section{Objectives} 
      
      In this thesis work the task that was se to be accomplished was to \textit{classify movements using kinect skeleton data}, this task was divided into several objectives that would help to accomplish it. The objectives are described as follows:

      \begin{enumerate}
         \item Visualize and label the Kinect skeleton data using 3D plots animations.
         \item Preprocess data to remove noise and outliers for better classification results.
         \item Analyze different approaches for handling the data, such as using raw data or applying Feature Engineering techniques.
         \item Implement and evaluate various Machine Learning models. 
         \item Conduct a comprehensive comparative analysis of the perfomance of the models based on evaluation metrics and execution time.
         \item Provide insights into the interpretability of selected models, aiding in the understanding of the approaches used by the models to classify movements.
      \end{enumerate}

\cleardoublepage