% ============================================== %
%
% TODO: 
%
%% ============================================== %
\chapter{Introduction}
   \section{Problem Statement}

   

   \section{Literature Review}
   
   \section{Dataset Overview}
   In this study, we utilized a Kinect Skeleton dataset, provided by the PsyComp Lab.,which plays a crucial role in the development of this research. This dataset is composed of recorded movements performed by a group of 22 individuals. The movements were performed in front of a Kinect sensor, which recorded the movements and saved them as a series of 3D coordinates. The dataset contains 10 different movements, each performed a various number of times by each individual. The movements are as follows:
   \begin{itemize}
      \item \textbf{Reach Overhead}: A movement involving the the subject rasing one of their arms above their head.
      \item \textbf{Chair to Chair}: A movement involving the subject sitting down on a chair, then standing up and sitting down on another chair.
      \item \textbf{Cross-Reach Left}: A movement involving the subject reaching their left arm across their body.
      \item \textbf{Reach Forward}: A movement involving the subject reaching their arm forward.
      \item \textbf{Hoop Walk}: A movement involving the subject walking in a circle.
      \item \textbf{Cross-Reach Right}: A movement similar to Cross-Reach Left, but involving the subject reaching their right arm across their body.
      \item \textbf{Right Leg Stand}: A movement involving the subject standing on their right leg.
      \item \textbf{Mat Walk}: A movement involving the subject walking on a mat.
      \item \textbf{Left Leg Stand}: A movement similar to Right Leg Stand, but involving the subject standing on their left leg.
      \item \textbf{TUG Walk}: The Timed Up and Go walk, a common test used to assess mobility and fall risk in elderly individuals.
   \end{itemize}

   \section{Aims and Objectives of the Study} 

\cleardoublepage