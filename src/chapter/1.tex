\hypersetup{colorlinks=true, linkcolor=blue, citecolor=red}

\chapter{Introduction} \label{chap:introduction}

   This thesis is structured in the following way: \textbf{Chapter} \ref{chap:introduction} presents the problem statement, literature review, dataset overview, aims, and objectives of the study. \textbf{Chapter} \ref{chap:dataset_analysis} presents the data collection methodology, data structure and attributes, patient's characteristics, movement visualization, and data processing. \textbf{Chapter} \ref{chap:methodology} presents the methodology used in this study, including the models, data splitting, and feature engineering approaches. \textbf{Chapter} \ref{chap:results_and_discussion} presents the evaluation metrics used, results obtained, and discussion of the results. Finally, \textbf{Chapter} \ref{chap:conclusions} presents conclusions of this study and future work.

   \section{Problem statement}

      Traditionally, movement classification requires high-quality sensors and complicated computer vision algorithms. However, with the arrival of the Microsoft Kinect sensor and the release of the Kinect SDK \cite{jana_kinect_2012}, it is now possible to obtain high-quality 3D skeleton data with a relatively low-cost device and with minimal effort. This opens up the possibility of using this data to classify movements performed by individuals, which can be used in a variety of applications, such as rehabilitation, sports, and fall risk assessment. In this thesis, the focus is on the latter, with the goal of using Kinect skeleton data to classify movements performed by elderly individuals.
  
   \section{Literature review}

      In recent times, detection and classification of human activity have found a wide range of applications in various fields. Among various sensors used, the Kinect sensor stands out for its affordability and ease of use. S.A. Abdul Shukor and Nor Asilah Saidin. conducted a study utilizing a Kinect sensor to detect human falls. Their system demonstrated accurate results \cite{saidin_analysis_2020}. 
      Tao Wang et al. conducted a study that involved gait analysis using a Kinect sensor for automatic and real-time detection of depression. The model developed achieved a classification accuracy of 93.75\% \cite{wang_gait_2021}.
      Naveen Kumar Mangal and Anil Kumar Tiwari conducted a study that developed a filter to improve the quality of three-dimensional coordinate data surrounding the body. With the aim of generating a movement signature crucial for kinematic analysis of musculoskeletal disorders. Findings from the study revealed that the range of motion values derived from the proposed filter significantly improved the monitoring accuracy of skeletal joints using a Kinect sensor \cite{mangal_kinect_2020}. 
      Shalini Nehra and Jagdish Lal Raheja created a human activity recognition system designed for indoor monitoring and detection of daily activities using a Kinect sensor. The results demonstrated the system's consistently high accuracy across various datasets, as reported in \cite{nehra_unobtrusive_2020}.
      Tan-Hsu Tan et al. conducted a study that developed a detection system to identify both daily and abnormal activities in elderly individuals. The performance was evaluated using a fourfold Cross-Validation approach, with precision at 95.5\%, recall at 95.6\%, specificity at 99.8\%, accuracy at 99.6\%, and an F1 score of 95.3\% \cite{tan_activity_2020}. 
      Weiyan Ren et al. used a Kinect sensor to gather posture data from twenty individuals while lying in bed. Data was then subjected to a Machine Learning approach using Support-Vector Machines architecture, resulting in a classification success rate of 97.1\% \cite{ren_human_2020}.
      \"{O}mer Faruk \.{I}nce et al. introduced an innovative biometric system designed to identify human activities within three-dimensional space. The study used the K-Nearest Neighbor algorithm as part of a Machine Learning approach for classification, achieving an accuracy of 86.1\% \cite{ince_human_2020}.
      Pramod Kumar Pisharady and Martin Saerbeck presented a multi-class algorithm for human posture detection and recognition. This algorithm remained invariant to changes in position and scale by leveraging geometric properties from Kinect data. Tested in both offline and real-time applications, it achieved a classification success of 95.78\% \cite{pisharady_kinect_2013}.
      Tao Wang et al. introduced a method for accurately distinguishing between various postures of five different individuals. A classification success rate exceeding 99\% was achieved \cite{wang_gait_2021}.


      \vspace{0.5cm}
      
      After reviewing the literature, a consistent trend was observed: the effectiveness of motion classification tends to reduce as the number of classes increases. Therefore, there is a recognized need for further research to improve classification performance in multi-class scenarios \cite{acis_classification_2023}. This study is specially focused on improving classification accuracy across 10 distinct classes.

   \section{Dataset overview}
      
      In this thesis, a dataset of Kinect skeleton data is used. Composed of recorded movements performed by a group of 22 individuals in front of a Kinect sensor, data is saved as a series of 3D coordinates. It contains 10 different movements, each performed a various number of times by each individual. The movements are listed in Table \ref{tab:movement_table}.

      \newpage

      \begin{table}[ht]
         \centering
         \begin{tabular}{@{}ccc@{}}
            \toprule
            \textbf{No.} & \textbf{Movement Name} & \textbf{Description} \\
            \midrule
            1 & Reach Overhead & \begin{tabular}[t]{@{}p{9cm}@{}} In a standing position, the subject raises one of their arms above their head.\end{tabular} \\
            
            2 & Chair to Chair & \begin{tabular}[t]{@{}p{9cm}@{}} Starting from a sitting position, the subject stands up, and then sits down on another chair.\end{tabular} \\
            
            3 & Cross-Reach Left & \begin{tabular}[t]{@{}p{9cm}@{}} In a standing position, the subject using their left arm reaches across their body to the right side.\end{tabular} \\
            
            4 & Cross-Reach Right & \begin{tabular}[t]{@{}p{9cm}@{}} In a standing position, the subject using their right arm reaches across their body to the left side.\end{tabular} \\
            
            5 & Reach Forward & \begin{tabular}[t]{@{}p{9cm}@{}} In a standing position, the subject reaches forward with one of their arms.\end{tabular} \\
            
            6 & Hoop Walk & \begin{tabular}[t]{@{}p{9cm}@{}} Starting from a standing position, the subject walks inside a hoop placed on the floor and then walks out of it.\end{tabular} \\
            
            7 & Right Leg Stand & \begin{tabular}[t]{@{}p{9cm}@{}} In a standing position, the subject raises their left leg and holds it in the air for a few seconds.\end{tabular} \\
            
            8 & Left Leg Stand & \begin{tabular}[t]{@{}p{9cm}@{}} In a standing position, the subject raises their right leg and holds it in the air for a few seconds.\end{tabular} \\
            
            9 & Mat Walk & \begin{tabular}[t]{@{}p{9cm}@{}} Starting from a standing position, the subject walks over a mat placed on the floor and then off it.\end{tabular} \\
            
            10 & TUG Walk & \begin{tabular}[t]{@{}p{9cm}@{}} Starting from a sitting position, the subject is asked to stand up, walk 3 meters, turn around, walk back to the chair, and sit down while being timed.\end{tabular} \\
            \bottomrule
         \end{tabular}
         \caption{Movements used in this study, along with a brief description.}
         \label{tab:movement_table}
   \end{table}

   \section{Objectives} 
      
      In this thesis work the task that is set to be accomplished is to \textit{classify movements using Kinect skeleton data}, this task is divided into several objectives that would help to accomplish it. Objectives are described as follows:

      \begin{enumerate}
         \item Visualize and label Kinect skeleton data using 3D plot animations.
         \item Pre-process data to remove noise and outliers for better classification results.
         \item Analyze different approaches for handling data, such as using raw data or applying Feature Engineering techniques.
         \item Implement and evaluate various Machine Learning models. 
         \item Conduct a comprehensive comparative analysis of the performance of the models based on evaluation metrics and execution time.
         \item Provide insights into the interpretability of selected models, aiding in understanding approaches used to classify movements.
      \end{enumerate}

\cleardoublepage