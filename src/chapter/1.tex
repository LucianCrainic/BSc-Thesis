% ============================================== %
%
% TODO: 
%
%% ============================================== %
\chapter{Introduction}

   % add todo notes
   
   \todo[inline]{Add a description of the thesis structure. This is a good place to start to explain briefly every chapter. Do this after you have written the rest of the thesis.}
   
   \section{Problem Statement}

   

   \section{Literature Review}
   
   \section{Dataset Overview}
   In this study, we utilized a Kinect Skeleton dataset, provided by the PsyComp Lab.,which plays a crucial role in the development of this research. This dataset is composed of recorded movements performed by a group of 22 individuals. The movements were performed in front of a Kinect sensor, which recorded the movements and saved them as a series of 3D coordinates. The dataset contains 10 different movements, each performed a various number of times by each individual. The movements are as follows:

   \begin{table}[ht]
      \centering
      \begin{tabular}{@{}ccc@{}}
          \toprule
          \textbf{No.} & \textbf{Movement Name} & \textbf{Description} \\
          \midrule
          1 & \textit{Reach Overhead} & \begin{tabular}[t]{@{}p{9cm}@{}} In a standing position, the subject raises one of their arms above their head.\end{tabular} \\
          
          2 & \textit {Chair to Chair} & \begin{tabular}[t]{@{}p{9cm}@{}} Starting from a sitting position, the subject stands up, then sits down on another chair.\end{tabular} \\
          
          3 & \textit{Cross-Reach Left} & \begin{tabular}[t]{@{}p{9cm}@{}} In a standing position, the subject using their left arm reaches across their body to the right side.\end{tabular} \\
          
          4 & \textit{Cross-Reach Right} & \begin{tabular}[t]{@{}p{9cm}@{}} Description 4 \end{tabular} \\
          
          5 & \textit{Reach Forward} & \begin{tabular}[t]{@{}p{9cm}@{}} Description 5 \end{tabular} \\
          
          6 & \textit{Hoop Walk} & \begin{tabular}[t]{@{}p{9cm}@{}} Description 6 \end{tabular} \\
          
          7 & \textit{Right Leg Stand} & \begin{tabular}[t]{@{}p{9cm}@{}} Description 7 \end{tabular} \\
          
          8 & \textit{Left Leg Stand} & \begin{tabular}[t]{@{}p{9cm}@{}} Description 8 \end{tabular} \\
          
          9 & \textit{Mat Walk} & \begin{tabular}[t]{@{}p{9cm}@{}} Description 9 \end{tabular} \\
          
          10 & \textit{TUG Walk} & \begin{tabular}[t]{@{}p{9cm}@{}} The Timed Up and Go Walk, a common test used to assess mobility and fall risk in elderly individuals.\end{tabular} \\
          \bottomrule
      \end{tabular}
      \caption{Movements used in this study, along with a brief description.}
      \label{tab:movement_table}
  \end{table}

   \section{Aims and Objectives of the Study} 
   
   \begin{enumerate}
      \item \textbf{Kinect Data Processing}: Process Kinect Skeleton Data from elderly individuals performing ten specific movements as part of the Fear of Falling assessment.
      \item TODO {}
      \item TODO
      \item TODO
   \end{enumerate}

\cleardoublepage