% ============================================== %
%
% ABSTRACT.tex
%
%% ============================================== %
\begin{abstract}
    This thesis conducts a detailed comparative study of several Machine Learning models, with a focus on their application to Kinect-based data for classifying human movements. The primary aim of this research is to evaluate these models to determine the most effective ones for accurately classifying movements recorded through Kinect sensors.\\

    This study begins with a introduction to Kinect technology, highlighting its ability to capture detailed movement data. Following this, an examination of a range of Machine Learning models, such as Support Vector Machines, Random Forest, Linear Regression, and so on. Each model is tested to evaluate its accuracy, processing efficiency, and robustness in accurately classifying various movements. 
    
    The core of this comparative analysis is a diverse dataset consisting of several movements captured through a Microsoft Kinect. The research methodology involves several steps: processing the Kinect data, extracting key features that are characteristic of specific movements, and applying the selected models to this improved data. Performance evaluation of each model using standard metrics like accuracy, precision, recall, and the F1 score, which provide a complete picture of their effectiveness. \\
    
    Over this study, valuable understandings are gained into the specific strengths and limitations of each model in the context of Kinect-based movement classification. The findings reveal that some models prove enhanced performance in certain situations, which is influenced by factors like the complexity of the captured movements and the characteristics of the dataset. 
    
    This thesis acts as a useful guide for researchers and professionals. It helps them pick the best models for similar work and sets the stage for more research in this area. This study contributes to the advancement of accurate and efficient Kinect-based data movement classification using Machine Learning methods, leading to more progress in this field.

\end{abstract}

% ! Uncomment this line when printing the thesis.
\let\cleardoublepage\clearpage