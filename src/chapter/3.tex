


\chapter{Methodology}
    \section{Data collection}
        \subsection{Microsoft Kinect}
            Microsoft Kinect is a line of motion sensing input devices that was 
            first released in 2010. The Kinect sensor consists of a RGB cameras, and 
            infrared projectos and detectors that allow it to measure depth. \cite{wikipedia_example}
        \subsection{Usage}

    \section{Data description and processing}
        In the following sections, we 

    \subsection{Filesystem structure}

    The dataset is structured in a filesystem-like structure, with the 
    root directory being the \texttt{Data} directory. The \texttt{Data} directory
    contains two subdirectories, \texttt{Patients} and \texttt{Questions}. 
    The \texttt{Patients} directory contains a subdirectory for each patient, 
    named \texttt{Patient-1}, \texttt{Patient-2}, etc. Each patient directory 
    contains a subdirectory for each measurement, named \texttt{M000}, \texttt{M001}, etc.
    Each measurement directory contains a subdirectory for each recording, 
    named \texttt{R000}, \texttt{R001}, etc. Each recording directory contains a 
    single file, named \texttt{file.csv}, which contains the data for that recording. 
    The \texttt{Questions} directory contains a single file, 
    named \texttt{Questions.xlsx}, which contains the questions asked to the patients.

    \begin{figure}[htbp]
        \centering
        \begin{forest}
            for tree={font=\sffamily,
            folder indent=.9em, folder icons,
            edge=densely dotted}
            [FoF-Dataset
              [Patients, this folder size=15pt
                  [Patient-1
                  [M000
                  [R000
                  [data.csv, is file]]]
                  ]
                  ]
              [Questions, this folder size=15pt
                  [Questions.xlxs, is file]]
            ]
        \end{forest}
        \caption{Filesystem structure}
    \end{figure}

    In the following example, the patient is named \texttt{Patient-1}, which means that 
    the patient ID is \texttt{1}. The patient ID is a number between 1 and 7. The dataset
    contains 7 patients. For simplicity, since all the patients follow the same structure,
    we will only show the structure of the first patient.
    
    \subsection{Patients data}

    \subsection{Questions data}
      
        \begin{table}[htbp]
            \centering
            \begin{tabular}{|p{2.5cm}|p{2.5cm}|p{2.5cm}|p{2.5cm}|p{2.5cm}|p{.5cm}|}
                \hline
                \textbf{Participant ID} & \textbf{Fallers/Non-fallers (5 past years)} & \textbf{Number of falls (past 2 years)} & \textbf{Was fall accidental?} & \textbf{Comments on the falls} & ... \\
                \hline
                Enter participant number & 0=NF 1=F & Enter a number if provided. Otherwise, leave blank & 0 = No fall 1 = Yes 2 = Not accidental &  & ... \\
                \hline
                01-300518 (example) & 1 & 1 & 1 & I fell over a kerb. I was in a hurry. & ... \\
                \hline
                ... & ... & ... & ... & ... & ... \\
                \hline
            \end{tabular}

            \caption{Questions.xlsx file structure example}
        \end{table}
    \subsection{Movements description}
        Each patient is required to perform a total of 9 movements. Each movement can be 
        perfomed once or multiple times depending on the researcher's instructions. 
        
        \begin{table}[htbp]
            \centering
            \begin{tabular}{|p{1cm}|p{3cm}|p{8cm}|}
                \hline
                \textbf{ID} & \textbf{Name} & \textbf{Description} \\
                \hline
                M000 & Chair to Chair & Sit on a chair, stand up, walk to another chair, sit down, stand up and walk back to the initial chair and sit down. \\
                \hline
                M001 & Hoop walk & Walk around a hoop. \\
                \hline
                M002 & Cross-reach left & Cross and reach with the left hand. \\
                \hline
                M003 & Cross-reach right & Cross and reach with the right hand. \\
                \hline
                M004 & Right leg stand & Stand on the right leg for an amount of time. \\
                \hline
                M005 & Left leg stand & Stand on the left leg for an amount of time. \\
                \hline
                M006 & Reach forward & Reach forward with one arm. \\
                \hline
                M007 & Reach overhead & Reach overhead with one hand. \\
                \hline
                M008 & Mat walk & Walk over a mat. \\
               \hline
                M009 & Tug walk & TODO. \\
                \hline
            \end{tabular}

            \caption{Questions.xlsx file structure example}
        \end{table}

        \textbf{ID} is the movement ID that is used in the dataset to identify the movement folder. 
        For example the folder \textbf{M000} contains the data for the movement \textit{Chair to Chair}.

        