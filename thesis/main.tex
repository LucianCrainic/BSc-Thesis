\documentclass[binding=0.6cm]{sapthesis}
\usepackage{microtype}
\usepackage[italian]{babel}
\usepackage[utf8]{inputenc}
\usepackage{hyperref}


\usepackage{listings}
\usepackage{xcolor}
 
\definecolor{codegreen}{rgb}{0,0.6,0}
\definecolor{blue}{rgb}{0.01, 0.28, 1.0}
\definecolor{codegray}{rgb}{0.5,0.5,0.5}
\definecolor{codepurple}{rgb}{0.58,0,0.82}
\definecolor{backcolour}{rgb}{0.95,0.95,0.92}

\lstdefinestyle{mystyle}{
    backgroundcolor=\color{white},   
    commentstyle=\color{codegreen},
    keywordstyle=\bfseries \color{blue},
    numberstyle=\color{codegray},
    stringstyle=\color{codepurple},
    basicstyle=\footnotesize,
    breakatwhitespace=false,         
    breaklines=true,                 
    captionpos=b,                    
    keepspaces=true,                 
    numbers=left,                    
    numbersep=5pt,                  
    showspaces=false,                
    showstringspaces=false,
    showtabs=false,                  
    tabsize=2
}

\lstset{style=mystyle}




\hypersetup{pdftitle={La mia tesi},pdfauthor={Lucian Dorin Crainic}}
\title{Data Analysis e Visualization dei movimenti di pazienti con paura di cadere}
\author{Lucian Dorin Crainic}
\IDnumber{1938430}
\course{Laurea Triennale in Informatica}
\courseorganizer{Facoltà di Ingegneria dell'Informazione, Informatica e Statistica}
\AcademicYear{2023/2024}
\advisor{Prof. Maurizio Mancini}
%\coadvisor{Dr. Sempronio}
\authoremail{crainic.lucian@gmail.com}
\copyyear{2022}
\thesistype{Tesi di Laurea Triennale}
\begin{document}
\lstset{language=Python}

% Frontespizio
\frontmatter
\maketitle

% Dedica 
\dedication{Audentes Fortuna iuvat \\ \emph{Virgilio, Eneide, X, 284}}

% Sommario 
% TODO: Lavorare sul sommario una volta finito la parte di codice. 
\begin{abstract}
In questo elaborato si vuole presentare un sistema di Data Analysis e Visualization 
per i movimenti di pazienti con paura di cadere. 
Il sistema è stato sviluppato in Python e utilizza le librerie Pandas, Numpy, Matplotlib e Seaborn. 
Il dataset utilizzato è stato fornito dal Prof. Maurizio Mancini e contiene i dati di 16 
pazienti che hanno effettuato un test di caduta. Il sistema è stato sviluppato in modo da essere 
facilmente estendibile e modificabile. 
\end{abstract}

% Indice
\tableofcontents

\mainmatter


\chapter{Introduzione}
    \section{Contesto e motivazione}
    \section{Problema di ricerca}
    \section{Obiettivi}
    \section{Domanda di ricerca}
    \section{Ambito e limitazioni}
    \section{Importanza dello studio}
\chapter{Revisione della letteratura}
    \section{Paura di cadere: definizione e prevalenza}
    \section{Fattori che contribuiscono alla paura di cadere}
    \section{Conseguenze della paura di cadere}
    \section{Approcci esistenti per valutare la paura di cadere}
    \section{Studi precedenti sull'analisi dei modelli di movimento nei pazienti con paura di cadere}
\chapter{Metodologia}
    \section{Raccolta dei dati}
    \section{Descrizione e pre-elaborazione del dataset}
    \section{Estrazione delle caratteristiche}
    \section{Tecniche di analisi statistica}
    \section{Modelli di apprendimento automatico utilizzati}
\chapter{Analisi e risultati}
    \section{Analisi descrittiva del dataset}
    \section{Confronto dei modelli di movimento tra i pazienti}
    \section{Analisi della correlazione tra paura di cadere e parametri di movimento}
    \section{Valutazione dei modelli di apprendimento automatico}
    \section{Discussione dei risultati}
\chapter{Discussione}
    \section{Interpretazione dei risultati}
    \section{Implicazioni delle scoperte per la comprensione della paura di cadere}
    \section{Limitazioni dello studio}
    \section{Raccomandazioni per futuri studi}
\chapter{Conclusioni}
    \section{Riassunto dello studio}
    \section{Contributi al campo}
    \section{Applicazioni pratiche}
    \section{Considerazioni finali}

\chapter*{Ringraziamenti}
\addcontentsline{toc}{chapter}{Ringraziamenti}
\begin{large}
Vorrei ringraziare, prima di tutto, il relatore della tesi \textbf{Prof. Angelo Spognardi} e il \textbf{Dr. Enrico Bassetti} per la disponibilità e per i consigli forniti durante l'attività di tirocinio. \\

Un ringraziamento anche a tutti i colleghi, diventati amici, che hanno condiviso con me questo percorso. \\

Infine ringrazio la mia famiglia per avermi aiutato a compiere questo importante passo e per essermi stata sempre vicina supportandomi in qualsiasi momento.
\end{large}


\backmatter
\phantomsection

\addcontentsline{toc}{chapter}{Bibliografia}
\bibliographystyle{sapthesis}
\bibliography{/home/Lucian/Github/Thesis/Thesis-LaTeX/thesis/src/references.bib}


\end{document}